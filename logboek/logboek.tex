%%%%%%%%%%%%%%%%%%%%%%%%%%%%%%
% LATEX-TEMPLATE GENERIEK
% Voor readme en meest recente versie, zie
% https://gitlab-fnwi.uva.nl/informatica/LaTeX-template.git
%%%%%%%%%%%%%%%%%%%%%%%%%%%%%%

%-------------------------------------------------------------------------------
%	PACKAGES EN DOCUMENT CONFIGURATIE
%-------------------------------------------------------------------------------

\documentclass{uva-inf-article}
\usepackage[dutch]{babel}
\usepackage{booktabs}
%-------------------------------------------------------------------------------
%	GEGEVENS VOOR IN DE TITEL, HEADER EN FOOTER
%-------------------------------------------------------------------------------

% Vul de naam van de opdracht in.
\assignment{MyHyvesBookPlusTagram}
% Vul het soort opdracht in.
\assignmenttype{Samenvatting}
% Vul de titel van de eindopdracht in.
\title{Logboek}

% Vul de volledige namen van alle auteurs in.
\authors{Niels Zwemmer}
% Vul de corresponderende UvAnetID's in.
\uvanetids{11025980}

% Vul altijd de naam in van diegene die het nakijkt, tutor of docent.
\tutor{Youri Voet}
% Vul indien nodig de naam van de begeleider in.
\mentor{}
% Vul eventueel ook de naam van de docent of vakcoordinator toe.
\docent{}
% Vul hier de naam van de PAV-groep  in.
\group{The Return Of MyHyvesBook+}
% Vul de naam van de cursus in.
\course{}
% Te vinden op onder andere Datanose.
\courseid{}
\date{\today}

% Dit is de datum die op het document komt te staan. Standaard is dat vandaag.

%-------------------------------------------------------------------------------
%	VOORPAGINA EN EVENTUEEL INHOUDSOPGAVE EN ABSTRACT
%-------------------------------------------------------------------------------

\begin{document}
\maketitle

%-------------------------------------------------------------------------------
%	INHOUD
%-------------------------------------------------------------------------------
\section{Algemeen}
\subsection{19-6-2017}
De dag verliep anders dan verwacht door een onverwachte opdracht die de TA's ons oplegde. Deze bestond uit een pitch van een paar minuten die ons plan en project 
duidelijk moesten maken. Dit plan werd goedgekeurd door Youri. Voor de pitches stond de tijd van 13:00 tot 14:00 ingesteld. Door deze pitch is onze vergadering verplaatst naar dinsdag 20-6-2017 van 11:00 tot 12:00.
Na de pitch hebben wij nog geprogrammeerd tot ongeveer 15:00. Daarna zijn wij naar huis gegaan en heeft iedereen voor zichzelf gewerkt.
\subsection{20-6-2017}
Allereerst zijn de laatste oefenopgaven nagekeken. Daarna zijn Felix en Marijn direct begonnen aan hun onderdelen implementeren waarna Paul en Niels zich bij hen voegden. Marijn heeft het logo ge\"updatet en is begonnen met de upload-class implementeren. Felix heeft een eerste versie van de CameraView gemaakt.
Paul is begonnen aan de profielpagina en Niels heeft hem daarmee geholpen. Ook heeft Niels de notulen voor de vergadering gemaakt die vandaag van 11:00 tot 12:00 plaatsvond. Tijdens deze vergadering zijn vooral een aantal ontwerpkeuzes besproken. Tot slot is er een template gemaakt voor de logboeken van ieder persoon, zodat productiviteit zo veel mogelijk ongehinderd kan blijven door het achteraf moeten stroomlijnen van dit soort zaken.
\subsection{21-6-2017}
De dag begon om 11:00 met een bijeenkomst van Marijn, Niels en Paul. Felix kon niet aanwezig zijn vandaag maar heeft dat gecompenseerd door veel thuis gewerkt te hebben aan de opdracht. Paul en Niels hebben de profielpagina nagenoeg afgemaakt en wachten nu tot Marijn en Felix hun eerste deel hebben ge\"implementeerd zodat zij verder kunnen. Dit zijn de onderdelen foto uploaden en wachtwoord wijzigen.\\
Om 13:00 ging Marijn naar zijn Minor Programmeren groep om daar TA te zijn. Paul en Niels hebben tussen 13:00 en 15:00 gewerkt aan de interface opleuken van de profielpagina naast de functionaliteiten die eerder al waren verwerkt.
Om 15:00 ging iedereen naar huis, met uitzondering van Marijn die pas om 16:00 klaar was.
\subsection{22-6-2017}
Vandaag hebben wij om 10 uur afgesproken om alvast een eerste versie van onze poster te maken. We zijn over het algemeen al tevreden over dat resultaat maar de definitieve versie zal minder tekst en meer, grotere plaatjes moeten bevatten.
Nadat de poster klaar was, hebben wij geprogrammeerd tot 15:00; de tijd waarop de PAV bijeenkomst begon.
Marijn was vanaf 13:00 weer te vinden in het Minor Programmeren lokaal. We hebben allemaal wat progressie geboekt. Paul heeft Felix ondersteunt met zijn camera implementatie waardoor Felix nu bijna toe is aan de filterimplementatie.
Marijn is bezig geweest aan de poster ontwerpen onder het toeziend oog van Felix als hoofd-design en Niels heeft een eerste implementatie gedaan voor het updaten van de profielfoto. Na de PAV-bijeenkomst is iedereen in de stromende regen naar huis gegaan.
\subsection{23-6-2017}
Lorem Ipsum Dolor sit amet.
\subsection{24-6-2017}
Lorem Ipsum Dolor sit amet.
\subsection{25-6-2017}
Lorem Ipsum Dolor sit amet.
\subsection{26-6-2017}
Lorem Ipsum Dolor sit amet.
\subsection{27-6-2017}
Lorem Ipsum Dolor sit amet.
\subsection{28-6-2017}
Lorem Ipsum Dolor sit amet.
\subsection{29-6-2017}
Lorem Ipsum Dolor sit amet.
\subsection{30-6-2017}
Lorem Ipsum Dolor sit amet.

\section{Persoonlijk}
\subsection{19-6-2017}
Lorem Ipsum Dolor sit amet.
\subsection{20-6-2017}
Lorem Ipsum Dolor sit amet.
\subsection{21-6-2017}
Lorem Ipsum Dolor sit amet.
\subsection{22-6-2017}
Lorem Ipsum Dolor sit amet.
\subsection{23-6-2017}
Lorem Ipsum Dolor sit amet.
\subsection{24-6-2017}
Lorem Ipsum Dolor sit amet.
\subsection{25-6-2017}
Lorem Ipsum Dolor sit amet.
\subsection{26-6-2017}
Lorem Ipsum Dolor sit amet.
\subsection{27-6-2017}
Lorem Ipsum Dolor sit amet.
\subsection{28-6-2017}
Lorem Ipsum Dolor sit amet.
\subsection{29-6-2017}
Lorem Ipsum Dolor sit amet.
\subsection{30-6-2017}
Lorem Ipsum Dolor sit amet.
\end{document}
