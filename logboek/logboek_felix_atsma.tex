%%%%%%%%%%%%%%%%%%%%%%%%%%%%%%
% LATEX-TEMPLATE GENERIEK
% Voor readme en meest recente versie, zie
% https://gitlab-fnwi.uva.nl/informatica/LaTeX-template.git
%%%%%%%%%%%%%%%%%%%%%%%%%%%%%%

%-------------------------------------------------------------------------------
%	PACKAGES EN DOCUMENT CONFIGURATIE
%-------------------------------------------------------------------------------

\documentclass{uva-inf-article}
\usepackage[dutch]{babel}
\usepackage{booktabs}
%-------------------------------------------------------------------------------
%	GEGEVENS VOOR IN DE TITEL, HEADER EN FOOTER
%-------------------------------------------------------------------------------

% Vul de naam van de opdracht in.
\assignment{MyHyvesBookPlusTagram}
% Vul het soort opdracht in.
\assignmenttype{Samenvatting}
% Vul de titel van de eindopdracht in.
\title{Logboek}

% Vul de volledige namen van alle auteurs in.
\authors{Felix Atsma}
% Vul de corresponderende UvAnetID's in.
\uvanetids{11035064}

% Vul altijd de naam in van diegene die het nakijkt, tutor of docent.
\tutor{Youri Voet}
% Vul indien nodig de naam van de begeleider in.
\mentor{}
% Vul eventueel ook de naam van de docent of vakcoordinator toe.
\docent{}
% Vul hier de naam van de PAV-groep  in.
\group{The Return Of MyHyvesBook+}
% Vul de naam van de cursus in.
\course{Multimedia}
% Te vinden op onder andere Datanose.
\courseid{}
\date{\today}

% Dit is de datum die op het document komt te staan. Standaard is dat vandaag.

%-------------------------------------------------------------------------------
%	VOORPAGINA EN EVENTUEEL INHOUDSOPGAVE EN ABSTRACT
%-------------------------------------------------------------------------------

\begin{document}
\maketitle

%-------------------------------------------------------------------------------
%	INHOUD
%-------------------------------------------------------------------------------
\section{Algemeen}
\subsection{19-6-2017}
Deze dag begon met het maken van een kleine presentatie van het projectplan
voor de TA, Youri. Het plan werdt goedgekeurd. Daarna zijn we aan de slag
gegaan met programmeren tot ongeveer 3 uur. Ik ben begonnen aan het maken van 
een preview voor de camera en heb daar thuis verder aan gewerkt.
\subsection{20-6-2017}
Na de opdrachten om 9 uur laten nakijken ben ik verder gegaan met de camera
view, hiermee was ik de hele dag bezig. 's Avonds was ik klaar met een simpele
preview, nog zonder de functionaliteit van foto's maken.
\subsection{21-6-2017}
Deze dag was ik niet aanwezig op het Science Park, dit komt doordat ik de
nacht ervoor laat door heb gewerkt, en ik had werk. Ondanks dit heb ik thuis
doorgewerkt. Het wisselen van voor- en achtercamera is afgemaakt, daarnaast is
er gewerkt aan bugfixes.
\subsection{22-6-2017}
Op deze dag heb ik het nemen van foto's geïmplementeerd en gewerkt aan de
layout van de camera view. Thuis heb ik ook nog het uploaden van foto's
werkend gekregen.
Naast het programmeren hebben ik en Marijn een eerste versie van de poster
gemaakt voor PAV.
\subsection{23-6-2017}
Ik heb me vooral bezig gehouden met het implementeren van de filters.
\subsection{24/25-6-2017}
Tijdens het weekend gewerkt aan wisselen tussen filters en het uploaden van de
gefilterde foto's.
\subsection{26-6-2017}
Gewerkt aan een comment functie, en er voor gezorgd dat de camera het hele
beeld opvult.
\subsection{27-6-2017}
Deze dag veel gewerkt aan problemen oplossen, met name het roteren van het
genomen plaatje, daarnaast een probleem met de comment box opgelost. Ook heb
ik nog snel nog een filter toegevoegd.
\subsection{28-6-2017}
Wederom weer problemen opgelost, nu de layout van posts op de timeline
aangepast en een bug met de front facing camera gefixt.
\subsection{29-6-2017}
Deze dag hebben we de laatste puntjes gezet, met de timeline layout
verbeteren, een crash verhelpen en comments aan de code toevoegen.
\subsection{30-6-2017}
Lorem Ipsum Dolor sit amet.

\end{document}
