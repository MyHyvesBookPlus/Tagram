%%%%%%%%%%%%%%%%%%%%%%%%%%%%%%
% LATEX-TEMPLATE GENERIEK
% Voor readme en meest recente versie, zie
% https://gitlab-fnwi.uva.nl/informatica/LaTeX-template.git
%%%%%%%%%%%%%%%%%%%%%%%%%%%%%%

%-------------------------------------------------------------------------------
%	PACKAGES EN DOCUMENT CONFIGURATIE
%-------------------------------------------------------------------------------

\documentclass{uva-inf-article}
\newcommand\tab[1][1cm]{\hspace*{#1}}
\newcommand\tabb[1][1.4cm]{\hspace*{#1}}
\usepackage[dutch]{babel}
\usepackage{booktabs}
%-------------------------------------------------------------------------------
%	GEGEVENS VOOR IN DE TITEL, HEADER EN FOOTER
%-------------------------------------------------------------------------------



% Vul de naam van de opdracht in.
\assignment{Notulen: The Return Of The MyHyvesBook+}
% Vul het soort opdracht in.
\assignmenttype{Samenvatting}
% Vul de titel van de eindopdracht in.
\title{Entry 1: Eerste offici\"ele vergadering}

% Vul de volledige namen van alle auteurs in.
\authors{Marijn Jansen; Felix Atsma; Paul Lagerweij; Niels Zwemmer}
% Vul de corresponderende UvAnetID's in.
\uvanetids{11166932; 11035064; 11306084; 11025980}

% Vul altijd de naam in van diegene die het nakijkt, tutor of docent.
\tutor{Robin Klusman}
% Vul eventueel ook de naam van de docent of vakcoordinator toe.
\docent{drs. A. van Inge}
% Vul hier de naam van de PAV-groep  in.
\group{C1 (C++)}
% Vul de naam van de cursus in.
\course{Multimedia}
% Te vinden op onder andere Datanose.
\courseid{5062MULT6Y}

% Dit is de datum die op het document komt te staan. Standaard is dat vandaag.
\date{\today}

%-------------------------------------------------------------------------------
%	VOORPAGINA EN EVENTUEEL INHOUDSOPGAVE EN ABSTRACT
%-------------------------------------------------------------------------------

\begin{document}
\maketitle

\noindent
\textbf{Data Vergadering}\\\\
Datum:\tab 20-6-2017
\\
Tijd:\tabb 11:00-12:00
\\\\
\noindent
Aanwezig (fysiek):\tabb\space\space\space Marijn, Felix, Niels, Paul\\
Aanwezig (telefonisch):\tab	-\\
Afwezig:\tabb\tabb\space\space\space\space-\\\\
\noindent
\textbf{Onderwerpen besproken}

\begin{itemize}
\item Naam gewijzigd:		MyHyvesBookPlusTagram (was MyHyvesBook+Stagram)
\item Implementatiekeuzes:		Besloten een API te gebruiken voor de feed in de stijl van Facebook die online is gevonden (indien nuttig).
\item Backend progressie besproken: 	Gaat de goede kant op.
\item Frontend progressie besproken:	Paul is nog wat aan het stoeien met de XML, maar schiet allemaal al aardig op.
\end{itemize}
\pagebreak

\noindent
\textbf{Positieve punten naar voren gekomen}\\\\
Het tempo zit er goed in en door geen gebrek aan motivatie en een goede wil om te leren, vergaat het werk tot nu toe zeer goed. De beginselen beginnen al zichtbaar te worden en steeds meer functionaliteit komt van de grond. Iedereen leeft goed zijn deadline na en komt op tijd voor de vergadering.
\\\\

\noindent
\textbf{Kort}
\begin{itemize}
\item Goed tempo
\item Backend gaat goed
\item Frontend gaat goed
\item We liggen op schema
\end{itemize}

\noindent
\textbf{Negatieve punten naar voren gekomen}\\\\
Paul en Niels hebben door een gebrek aan ervaring nog wat moeite om de profilepage op te zetten. De oplossing daarvoor is meer hulp vragen aan Felix en Marijn en actief op internet zoeken naar bruikbare guides. Daarnaast is het implementeren van de CameraView in eenvoud tegengevallen waardoor daar wellicht meer ondersteuning voor nodig is door de rest van het team.
\\\\

\noindent
\textbf{Kort}
\begin{itemize}
\item Gebrek ervaring Paul en Niels
\item Meer hulp vragen aan teamgenoten nodig
\item Tegenvaller tijdens implementatie van de CameraView

\end{itemize}

								\vfill	\hfill	\textit{“As `t net kin sa`t moat, dan moat `t mar sa`t kin.”}

%-------------------------------------------------------------------------------
%	INHOUD
%-------------------------------------------------------------------------------

\end{document}
