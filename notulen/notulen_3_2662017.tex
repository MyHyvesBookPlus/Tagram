%%%%%%%%%%%%%%%%%%%%%%%%%%%%%%
% LATEX-TEMPLATE GENERIEK
% Voor readme en meest recente versie, zie
% https://gitlab-fnwi.uva.nl/informatica/LaTeX-template.git
%%%%%%%%%%%%%%%%%%%%%%%%%%%%%%

%-------------------------------------------------------------------------------
%	PACKAGES EN DOCUMENT CONFIGURATIE
%-------------------------------------------------------------------------------

\documentclass{uva-inf-article}
\newcommand\tab[1][1cm]{\hspace*{#1}}
\newcommand\tabb[1][1.4cm]{\hspace*{#1}}
\usepackage[dutch]{babel}
\usepackage{booktabs}
%-------------------------------------------------------------------------------
%	GEGEVENS VOOR IN DE TITEL, HEADER EN FOOTER
%-------------------------------------------------------------------------------



% Vul de naam van de opdracht in.
\assignment{Notulen: The Return Of The MyHyvesBook+}
% Vul het soort opdracht in.
\assignmenttype{Samenvatting}
% Vul de titel van de eindopdracht in.
\title{Entry 3: Afrondende fase}

% Vul de volledige namen van alle auteurs in.
\authors{Marijn Jansen; Felix Atsma; Paul Lagerweij; Niels Zwemmer}
% Vul de corresponderende UvAnetID's in.
\uvanetids{11166932; 11035064; 11306084; 11025980}

% Vul altijd de naam in van diegene die het nakijkt, tutor of docent.
\tutor{Robin Klusman}
% Vul eventueel ook de naam van de docent of vakcoordinator toe.
\docent{drs. A. van Inge}
% Vul hier de naam van de PAV-groep  in.
\group{C1 (C++)}
% Vul de naam van de cursus in.
\course{Multimedia}
% Te vinden op onder andere Datanose.
\courseid{5062MULT6Y}

% Dit is de datum die op het document komt te staan. Standaard is dat vandaag.
\date{\today}

%-------------------------------------------------------------------------------
%	VOORPAGINA EN EVENTUEEL INHOUDSOPGAVE EN ABSTRACT
%-------------------------------------------------------------------------------

\begin{document}
\maketitle

\noindent
\textbf{Data Vergadering}\\\\
Datum:\tab 26-6-2017
\\
Tijd:\tabb 11:00-12:00
\\
\\
\noindent
Aanwezig (fysiek):\tabb\space\space\space Marijn, Felix, Niels\\
Aanwezig (telefonisch):\tab Paul\\
Afwezig:\tabb\tabb\space\space\space\space-\\\\
\noindent
\textbf{Onderwerpen besproken}

\begin{itemize}
\item Implementatiekeuzes:		Besluit genomen de profielfoto beter te maken door file te uploaden i.p.v. thumbnail-bitmap.
\item Evaluatie:		Hoe staan we ervoor? De app krijgt nu vorm. Eerste poging tot alles mergen zal morgen zijn.
\item Evaluatie: 	Teksten van de poster en flyers zijn nu grotendeels bedacht. Marijn en Felix zullen gezamelijk de layout verder verzorgen.
\end{itemize}
\pagebreak

\noindent
\textbf{Positieve punten naar voren gekomen}\\\\
Onze app krijgt nu vorm in de zin dat er steeds meer functionaliteit op elkaar afgestemd wordt en samenwerkt. De losse modules implementeren zoals wij hadden bedacht is een geslaagd idee. Niels heeft zijn lastige bug nu opgelost waardoor hogere resolutie foto's kunnen worden opgeslagen. Felix is bijna klaar met de filters. Marijn zal aan het einde van de middag de verschillende upload-classes en de download-class klaar hebben. Paul werkt als groepswerker op alle fronten mee aan de code.
\\\\

\noindent
\textbf{Kort}
\begin{itemize}
\item Wederom goede vooruitgang, zowel poster als programma.
\item File uploaden is nu geslaagd en profielfoto is niet meer lelijk.
\item We liggen op schema; back-end bijna af.
\end{itemize}

\noindent
\textbf{Negatieve punten naar voren gekomen}\\\\
De bugs oplossen heeft meer tijd gekost dan gehoopt. Daardoor leken we wat van het schema af te zullen wijken. Gelukkig bleek dit mee te vallen. Echter zal er toch nog even hard doorgewerkt moeten worden de laatste paar dagen.\\
Er bestaat nog steeds veel onduidelijkheid over het individuele verslag en de overige PAV onderdelen. Communicatie bij het vak laat wederom te wensen over. Niels zal een e-mail sturen naar Toto en hopelijk kan er morgen gesproken worden met Youri over een aantal zaken.
\\\\

\noindent
\textbf{Kort}
\begin{itemize}
\item Bugs oplossen heeft meer tijd gekost dan gehoopt.
\item Veel onduidelijkheid over het individuele verslag.
\item Communicatie laat te wensen over.
\item Er is nog wat werk aan de winkel, niet alles werkt nog naar behoren.
\end{itemize}

								\vfill	\hfill	\textit{“Ik sis 't mar gewoan: ik ha in hekel oan moandeitemoarn.”}

%-------------------------------------------------------------------------------
%	INHOUD
%-------------------------------------------------------------------------------

\end{document}
