%%%%%%%%%%%%%%%%%%%%%%%%%%%%%%
% LATEX-TEMPLATE GENERIEK
% Voor readme en meest recente versie, zie
% https://gitlab-fnwi.uva.nl/informatica/LaTeX-template.git
%%%%%%%%%%%%%%%%%%%%%%%%%%%%%%

%-------------------------------------------------------------------------------
%	PACKAGES EN DOCUMENT CONFIGURATIE
%-------------------------------------------------------------------------------

\documentclass{uva-inf-article}
\newcommand\tab[1][1cm]{\hspace*{#1}}
\newcommand\tabb[1][1.4cm]{\hspace*{#1}}
\usepackage[dutch]{babel}
\usepackage{booktabs}
%-------------------------------------------------------------------------------
%	GEGEVENS VOOR IN DE TITEL, HEADER EN FOOTER
%-------------------------------------------------------------------------------



% Vul de naam van de opdracht in.
\assignment{Notulen: The Return Of The MyHyvesBook+}
% Vul het soort opdracht in.
\assignmenttype{Samenvatting}
% Vul de titel van de eindopdracht in.
\title{Entry 2: Nu komt het echte werk}

% Vul de volledige namen van alle auteurs in.
\authors{Marijn Jansen; Felix Atsma; Paul Lagerweij; Niels Zwemmer}
% Vul de corresponderende UvAnetID's in.
\uvanetids{11166932; 11035064; 11306084; 11025980}

% Vul altijd de naam in van diegene die het nakijkt, tutor of docent.
\tutor{Robin Klusman}
% Vul eventueel ook de naam van de docent of vakcoordinator toe.
\docent{drs. A. van Inge}
% Vul hier de naam van de PAV-groep  in.
\group{C1 (C++)}
% Vul de naam van de cursus in.
\course{Multimedia}
% Te vinden op onder andere Datanose.
\courseid{5062MULT6Y}

% Dit is de datum die op het document komt te staan. Standaard is dat vandaag.
\date{\today}

%-------------------------------------------------------------------------------
%	VOORPAGINA EN EVENTUEEL INHOUDSOPGAVE EN ABSTRACT
%-------------------------------------------------------------------------------

\begin{document}
\maketitle

\noindent
\textbf{Data Vergadering}\\\\
Datum:\tab 23-6-2017
\\
Tijd:\tabb 11:30-12:00
\\
Noot: De vergadering is iets later begonnen door een vertraging van zowel Niels als Felix.
\\\\
\noindent
Aanwezig (fysiek):\tabb\space\space\space Marijn, Felix, Niels\\
Aanwezig (telefonisch):\tab-\\
Afwezig:\tabb\tabb\space\space\space\space Paul\\\\
\noindent
\textbf{Onderwerpen besproken}

\begin{itemize}
\item Implementatiekeuzes:		Ingebouwde camera voor de profielfoto, mogelijkheid tot uitbreiden met gezichtsherkenning (indien tijd).
\item Evaluatie:		Hoe staan we ervoor? Back-end grotendeels af.
\item Problemen aan de kant van Multimedia-team: 	Niet ingedeeld in een groep, projectplan niet her-inleverbaar.
\item Verdeling aangepast:		Iedereen helpt elkaar met methodes waar zij verstand van hebben.
\end{itemize}
\pagebreak

\noindent
\textbf{Positieve punten naar voren gekomen}\\\\
We maken goede vooruitgang waarbij we zowel de code als de poster in gedachten hebben. De code ziet er netjes uit en iedereen houdt een consistente stijl aan. Dit zorgt voor een goede werksfeer en weinig over en weer uitleg nodig.
Het gebruik van de ingebouwde camera voor een profielfoto (aangezien daar weinig verdere methodes voor nodig zijn zoals die nodig zijn voor een post plaatsen), zorgt ervoor dat veel werk uit handen genomen wordt en wij onszelf meer tijd gunnen voor de overige benodigde functionaliteit. We zitten goed op schema en lopen zelfs op punten wat voor. Dit moeten we zo volhouden om te zorgen dat de app ook nog daadwerkelijk volgemaakt kan worden met posts voor de oplevering volgende week.
De back-end is nu grotendeels af, wat ervoor zorgt dat we nu ook echt de filters en andere algoritmes kunnen implementeren en testen.
\\\\

\noindent
\textbf{Kort}
\begin{itemize}
\item Goede vooruitgang, zowel poster als programma.
\item Ingebouwde camera goede keuze geweest.
\item We liggen op schema; back-end bijna af.
\end{itemize}

\noindent
\textbf{Negatieve punten naar voren gekomen}\\\\
Paul heeft wat tijd nodig om te herstellen na een aantal dagen achter elkaar aanwezig te zijn geweest op de UvA. Thuis werkt hij echter wel door.
Daarnaast zijn we niet goed ingedeeld in een groep op Blackboard, hiervoor zal Niels een mailtje sturen naar Toto om te vragen voor ondersteuning.
\\\\

\noindent
\textbf{Kort}
\begin{itemize}
\item Paul moet even herstellen.
\item Geen groep op Blackboard.

\end{itemize}

								\vfill	\hfill	\textit{“Wer hedden d’r zin oan!”}

%-------------------------------------------------------------------------------
%	INHOUD
%-------------------------------------------------------------------------------

\end{document}
