%%%%%%%%%%%%%%%%%%%%%%%%%%%%%%
% LATEX-TEMPLATE GENERIEK
% Voor readme en meest recente versie, zie
% https://gitlab-fnwi.uva.nl/informatica/LaTeX-template.git
%%%%%%%%%%%%%%%%%%%%%%%%%%%%%%

%-------------------------------------------------------------------------------
%	PACKAGES EN DOCUMENT CONFIGURATIE
%-------------------------------------------------------------------------------

\documentclass{uva-inf-article}
\newcommand\tab[1][1cm]{\hspace*{#1}}
\newcommand\tabb[1][1.4cm]{\hspace*{#1}}
\usepackage[dutch]{babel}
\usepackage{booktabs}
%-------------------------------------------------------------------------------
%	GEGEVENS VOOR IN DE TITEL, HEADER EN FOOTER
%-------------------------------------------------------------------------------



% Vul de naam van de opdracht in.
\assignment{Notulen: The Return Of The MyHyvesBook+}
% Vul het soort opdracht in.
\assignmenttype{Samenvatting}
% Vul de titel van de eindopdracht in.
\title{Entry 0: Eerste opzet project}

% Vul de volledige namen van alle auteurs in.
\authors{Marijn Jansen; Felix Atsma; Paul Lagerweij; Niels Zwemmer}
% Vul de corresponderende UvAnetID's in.
\uvanetids{11166932; 11035064; 11306084; 11025980}

% Vul altijd de naam in van diegene die het nakijkt, tutor of docent.
\tutor{Robin Klusman}
% Vul eventueel ook de naam van de docent of vakcoordinator toe.
\docent{drs. A. van Inge}
% Vul hier de naam van de PAV-groep  in.
\group{C1 (C++)}
% Vul de naam van de cursus in.
\course{Multimedia}
% Te vinden op onder andere Datanose.
\courseid{5062MULT6Y}

% Dit is de datum die op het document komt te staan. Standaard is dat vandaag.
\date{\today}

%-------------------------------------------------------------------------------
%	VOORPAGINA EN EVENTUEEL INHOUDSOPGAVE EN ABSTRACT
%-------------------------------------------------------------------------------

\begin{document}
\maketitle

\noindent
\textbf{Data Vergadering}\\\\
Datum:\tab 13-6-2017
\\
Tijd:\tabb 10:30-12:00
\\\\
\noindent
Aanwezig (fysiek):\tabb\space\space\space Marijn, Felix, Niels\\
Aanwezig (telefonisch):\tab	Paul\\
Afwezig:\tabb\tabb\space\space\space\space-\\\\
\noindent
\textbf{Onderwerpen besproken}

\begin{itemize}
\item Naam bedacht:		MyHyvesBook+Stagram
\item Concept bedacht:		Een soort live-feed maken die openbaar zichtbaar is voor alle leden van MyHyvesBook+Stagram.
\item Projectplan opzetten: 	Taak voor Niels
\item Verdeling gemaakt:		Taakverdelingen zijn terug te vinden in het projectplan
\item Implementatiekeuzes:	We gaan werken met Firebase op verzoek van Marijn. 
\end{itemize}
\pagebreak

\noindent
\textbf{Positieve punten naar voren gekomen}\\\\
We hebben nu een concept en naam bedacht. We gaan werken met firebase en willen realistische doelstellingen aannemen. We gaan zo snel mogelijk de backend opzetten en zorgen dat het esthetische gedeelte draait, daarna gaan we over op het multimedia gedeelte.
Niels gaat het projectplan opzetten in overleg met de andere groepsleden via de WhatsApp-groep en stuurt de uiteindelijk versie rond ter nakijken. Zodra deze is goedgekeurd door alle leden, wordt het projectplan z.s.m. ingeleverd.
Marijn heeft het meest gewerkt met app-development (iOS) en weet dus aardig wat concepten binnen deze wereld. Hij zal dan ook de groepsvoorzitter zijn en nauw samenwerken met alle leden uit de groep en waar nodig ondersteuning kunnen bieden op theoretisch vlak.
\\\\

\noindent
\textbf{Kort}
\begin{itemize}
\item Concept en naam bedacht
\item Backend z.s.m.
\item Projectplan z.s.m.
\item Marijn ervaring app development
\end{itemize}

\noindent
\textbf{Negatieve punten naar voren gekomen}\\\\
Paul kan minder vaak aanwezig zijn bij de vergaderingen of tijdens programmeersessies op de UvA door ziekte. Hierover zal de studieadviseur op de hoogte worden gebracht. We verwachten hier niet al te veel problemen mee te krijgen.
We zijn een groep met weinig ervaring wat betreft Android Programming. De inleidende opdrachten zullen echter voldoende zijn om ons op gang te helpen en dus zal iedereen zijn steentje kunnen bijdragen.
Felix en Marijn hebben een herkansing voor een ander vak die zij dienen te leren. Hierdoor zullen zij mogelijk tijdens de laatste paar dagen minder uren per dag kunnen besteden aan het project. Dit kunnen wij met elkaar opvangen door Felix’ en Marijn’s werkdruk iets te verdelen onder de rest van de groep.
\\\\

\noindent
\textbf{Kort}
\begin{itemize}
\item Gegronde omstandigheden Paul
\item Weinig Android ervaring
\item Herkansing Felix en Marijn

\end{itemize}

								\vfill	\hfill	\textit{“Wer hedden d’r zin oan!”}

%-------------------------------------------------------------------------------
%	INHOUD
%-------------------------------------------------------------------------------

\end{document}
